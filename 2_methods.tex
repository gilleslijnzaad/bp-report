% !TEX root = thesis.tex

\section{Methods}\label{sec:methods}
Following~\cite{Unni2017} an \(n\)-back task with five levels (\(n = 0,1,2,3,4\)) was used to manipulate working memory load. 
This task was integrated into the driving task by means of speed regulation. 
The designed environment also had an additional working memory constraint: a construction site with narrower lanes.  

\subsection{Participants}
A total of 38 volunteers (23 male, 12 female, 3 other) aged 20--36 (\(M = 23.1 \pm 3.0\)), possessing a category B driver's license, participated in this experiment. 
All participants signed an informed consent form prior to the experiment and were compensated 12 euros for their participation.

\subsection{Experimental Set-up}
The experiment took place on a simulated three-lane highway with no major turns/bends in the road. 
The features of the environment were minimal. 
Either side of the road was coloured green, signifying grass. 
There were no median strips dividing the road from the rest of the environment. 
Apart from a single other car, represented by a blue rectangle and referred to as the \textit{autocar}, there were no other objects/traffic on the highway. 
The autocar would stick to traffic rules such as overtaking from the left, staying on the right lane as much as possible, and following the current speed limit. 

The participant could see a black dashboard that filled the bottom of the screen. 
Here the speed of the car was visible (as an integer). 
When the left or right indicators were pressed, they would appear on the dashboard in the respective sides as yellow blinking arrows. 
The simulation had three rear-view mirrors: one on the top, one on the left, and one on the right. 
The autocar was visible in the corresponding mirrors depending on the distance from the car. 

In the construction condition the leftmost lane is closed off by a continuous row of pylons. 
The lanes were separated by a full yellow line and were narrower than the non-construction condition. 

Speed signs that passed were identical to general speed signs in The Netherlands; black digits enclosed by a red circle. 

% explain n-back task
Within each trial the participants were presented with at least nine speed signs at intervals of 20 seconds, with the exception of the first speed sign appearing after 5 seconds. 
For \(n\)-back tasks with \(n > 0\), there was a build-up phase of \(n\) speed signs preceding the nine speed signs where the participant would perform the task. 
For example, for \(n = 4\), the build-up phase would be the first four speed signs. 
After the build-up phase, the task of regulating speed would start. 
Due to a difference in length of build-up phases per \(n\)-back trial, each trial differs in number of speed signs, as well time taken.

% equipment and stuff
Participants interacted with the simulation using a steering wheel with blinkers and a throttle and brake pedal (Driving Force GT by Logitech). 
The steering wheel was secured to the table in front of the screen and remained in the same location for all participants. 
The pedals were placed on the floor such that participants could move it closer or further depending on their level of comfort. 
An eye-tracking camera (EyeLink Portable Duo, SR Research, Missisauga, Canada), placed between the screen and the steering wheel, was used to continuously record the eye movements and pupil size of participants. 

\subsection{Experimental Procedure}
The procedure of the current experiment follows that of \citet{Scheunemann2019}.
The experiment consisted of 20 trials in total, divided by a short break into two blocks of 10 trials each. 
Within a block, each \(n\)-back trial appeared twice: once with a construction site and once without. 
The order of the trials was determined pseudorandomly with a few conditions. 
Firstly, no \(n\)-back level could appear twice in a row. 
Secondly, the construction/non-construction conditions were alternated from trial to trial. 
These constraints on the randomization were incorporated with the aim of avoiding habituation effects for the memory task and the visuospatial demands. 
Finally, the order of the trials in the first block was reversed to form the order of trials in the second block.

Prior to performing the experiment the participant was given instructions about the driving and the memory task. 
They then performed a practice round (one 2-back trial with no construction and a total of 5 speed signs) to get accustomed to the simulation and the steering wheel. 
Next, the eye-tracker was calibrated. 
This involved the participant following a target around the computer screen with their eyes. 
This procedure was repeated twice: once to calibrate and once to validate whether that calibration was accurate. 
Calibration was performed again in the case that the validation was inaccurate.

After calibration, the experiment began. 
Before every trial a pop-up message appeared telling the participant which \(n\)-back task they should perform in the next trial. 
The percentage of total trials they had already completed was also shown in the message. 
Furthermore, every trial (excluding the very first one) was preceded by an eye-tracking drift correction. 
This required the participant to look at a target at the center of the screen. 
If the measured eye position deviated too far from the position of the target, calibration was performed again. 
Otherwise the deviation was automatically taken into account with recording of the eye position. 

\subsection{Data collection}
A number of different variables pertaining to driving behavior were recorded at a rate of 200 Hz. 
The use of the accelerator and brake pedals was recorded as numbers ranging from 0 (not pressed) to 1 (fully pressed). 
The angle of the steering wheel was recorded as a number ranging from -1 (left) to 1 (right).
In order to measure lane centering, the position and orientation of the participant's car were recorded as [measure]. 
Finally, the speed of the participant's car was recorded along with the occurring speed signs to calculate \(n\)-back task performance. 

The eye-tracker recorded a number of raw variables at a rate of 500 Hz, two of which are relevant for the current study.
Eye positions are measured in \(x\) and \(y\) coordinates relative to the PC monitor (\(1920 \times 1080\) px).
Pupil dilation was measured in terms of diameter in arbitrary units. 

\subsection{Data analysis}
