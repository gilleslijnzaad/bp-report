% !TEX root = thesis.tex

\begin{abstract}
Over the past decade interest in adaptive driving has increased. 
In adaptive driving, control over the vehicle is dynamically divided between the driver and an intelligent system. 
Particularly interesting is the idea of systems that adapt their degree of control to the cognitive load that the driver is experiencing. 
A system that is able to counteract the negative effects of a mental overload can greatly benefit the driver's safety.
In order to develop this, a robust method of measuring an individual's cognitive load must first be found.
This study focuses on pupillometry as a possible predictor for cognitive load, which is here defined as a combination of working memory load (WML) and visuospatial demands. We expected to find a positive correlation between cognitive load and pupil size.
Additionally, we were interested in the effect of cognitive load on speed-keeping efforts, as measured by eye fixations on the speedometer. 
We expected to find a negative correlation between cognitive load and speedometer checking.

To investigate this, a simulated-driving experiment with eye-tracking was conducted in which WML and visuospatial demands were manipulated separately. 
In the simulation participants drove on a straight highway for 60 minutes. 
WML was manipulated by an \(n\)-back task (\(n = 0,1,2,3,4\)), performed by means of speed regulation. 
Visuospatial demands were manipulated by a change in the driving environment: a construction site with reduced lane width, increasing driving difficulty. 

Results indicate that pupil size is a predictor for WML, but not for visuospatial demands. Additional research is required to consolidate whether visuospatial demands indeed have no effect on pupil size. If so, the utility of pupillometry as a predictor for cognitive load is challenged. 
Moreover, a negative correlation between WML and number of fixations on the speedometer was found. This highlights speed-keeping aid as an application for adaptive automation based on cognitive load.

\vspace{6ex}
\end{abstract}
