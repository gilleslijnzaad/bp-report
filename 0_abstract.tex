% !TEX root = thesis.tex

\begin{abstract}
In adaptive driving, control over the vehicle is dynamically divided between the driver and an intelligent system. 
In order to develop a system that adapts its degree of control to the mental state of the driver, a robust method of measuring their cognitive load is required.
This study focuses on pupillometry as a possible predictor for cognitive load, which is here defined as a combination of working memory load (WML) and visuospatial demands. 
We expected to find a positive correlation between cognitive load and pupil size.
Additionally, we were interested in the effect of cognitive load on speed-keeping efforts, as measured by eye fixations on the speedometer. 
We expected to find a negative correlation between cognitive load and speedometer checking.

To investigate this, a simulated-driving experiment with eye-tracking was conducted in which WML and visuospatial demands were manipulated separately. 
In the simulation participants drove on a straight highway for 60 minutes. 
WML was manipulated by an \(n\)-back task (\(n = 0,1,2,3,4\)), performed by means of speed regulation. 
Visuospatial demands were manipulated by a change in the driving environment: a construction site with reduced lane width, increasing driving difficulty. 

Results indicate that pupil size is a predictor for WML, but not for visuospatial demands. 
We conclude that in order to fully capture cognitive load while driving, pupillometry should be used in combination with a measure of visuospatial demands.
Moreover, a negative correlation between WML and number of fixations on the speedometer was found. 
This highlights speed-keeping aid as an application for adaptive automation based on cognitive load.

\vspace{6ex}
\end{abstract}