% !TEX root = thesis.tex

\section{Introduction}\label{sec:introduction}
This is a document that helps you writing your Bachelor’s thesis.
It has tips about academic writing style, but also about the style in which you have to write your thesis.
A Bachelor’s thesis has a length of 10–15 pages (15 pages is the maximum).
If you have more to discuss or show, you can put these additional materials in the Appendices at the end of the thesis.
The appendices are not counted in the number of pages.

It is important that Bachelor’s theses have a uniform style so that its layout is predictable.
It is also a good way to practise using a style defined by someone else.
When you write papers for journals and conferences, you will have to keep to a particular style.
In many cases, reports for companies and government organizations are also required to be in a particular format.

The style has two elements: the form and the content.
A standard/uniform design ensures that reports can be easily compared and that the reader does not have to ‘get used to’ a new layout each time.
A well-designed layout also makes the text easier to read.
Uniformity of content enables the reader to find things easily in the text, and helps inexperienced writers to organize their material.

These instructions should be followed as closely as possible, as far as the word processing program allows.
This will enable you to fit 600-700 words on a page.
This document was created in \LaTeX and may be used as a template.
It also serves as a reference for design and layout.
