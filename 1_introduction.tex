% !TEX root = thesis.tex

\section{Introduction}\label{sec:introduction}

% ----- GENERAL EXPLANATIONS -----
% driving
Driving a car is a challenging task. 
It involves processing a large amount of stimuli and constantly updating a mental model of the environment.
Not to mention, operating a vehicle requires making appropriate decisions to ensure the safety of both the driver and other road users.
Driving is even more challenging for young and novice drivers. 
They are more prone to a high level of mental workload than experienced drivers due to their low level of operating skills, lack of driving experience \citep{Gregersen1996} and not fully-matured prefrontal cortex \citep{Ross2014}.
This in turn is one of the causes for the relatively large number of traffic accidents that young drivers are involved in \citep{Sena2013}.
And indeed, more generally, human failure is the cause of the majority of traffic accidents \citep{DeWaard1996}.

% automated driving
An often proposed solution to this issue is automated driving, i.e.\ the vehicle being operated by an intelligent system \citep{Cabrall2018}.
However, \citet{Brookhuis2007} state that human supervision is necessary even in fully automated driving to handle abnormal situations.
Situations where the human operator must suddenly take back manual control then pose a serious risk:
the driver is likely to respond inadequately due to their reduced attentional awareness and the erosion of their operating skills \citep{Dijksterhuis2012}.
This is where adaptive automation comes into play.

% adaptive automation
In adaptive automation the division of control between the machine and the human operator is not static.
Rather, it is based on changes in the physical environment or the condition of the operator \citep{Sheridan2011}.
An important facet of this is adaptive automation based on the human factor of mental workload.
An intelligent car that counteracts the negative effects of a high cognitive load on driving performance can greatly benefit the driver's safety.

% cognitive load
We must first ask what the concept of cognitive load means.
Let us define cognitive load as \textit{the level of perceived cognitive effort when performing a task}.
Two elements to cognitive load are most important in the context of driving: central and visual demands \citep{DeWaard1996}. 

% explain WML as factor to cognitive load
Central demands have to do with working memory load.
In driving, working memory plays an important role in remaining focused on the task at hand, i.e.\ maintaining cognitive control \citep{Wood2016}.
It is also important in maintaining task goals, whether high-level (e.g.\ planning a route) or low-level (e.g.\ planning an overtaking manoeuvre).

% explain visuospatial demands factor to cognitive load
The task of driving has some intricate visual demands as well.
Visuospatial attention is required to process the movement of objects in traffic such as cars, pedestrians and traffic signs \citep{Zheng2020}.
This is a reflected by a number of studies on road accidents, linking reduced visuospatial attentional abilities (due to for example old age) to deteriorating driving performance [for a review see \citet{Owsley2010}].
In practice, the requirement of visuospatial attention means that drivers must continuously scan their environment since criticial visual events can occur anywhere at any time.

% measuring cognitive load
In order to develop a system that adapts to the driver's cognitive load it is essential to find a robust method of measuring cognitive load.
Changes in an individual's cognitive load are reflected by a number of physiological measures including heart rate variability, brainwave levels (as measured by an electroencephalogram; EEG), skin galvanic response and pupillary response \citep{Haapalainen2010}.
The current study focuses on the pupillary response as a possible predictor for cognitive load.

We are also interested in the effect of cognitive load on speed-keeping.
A relationship between cognitive load and speed-keeping performance would reveal that speed-keeping is a necessary application of adaptive automation. 
In other words, if a high cognitive load leads to diminished speed-keeping performance, then adaptive automation should aid the driver with keeping their speed.
In this study we will use eye fixations on the speedometer as a measure of speed-keeping efforts. 

% ----- RESEARCH QUESTIONS -----
This study aims to investigate three questions concerning eye-tracking measurements and cognitive load while driving.
\textbf{(1)} Can the current level of working memory load (WML) be predicted by pupil size?
\textbf{(2)} Is there an interaction between WML and visuospatial demands on pupil size?
\textbf{(3)} Does the number of eye fixations on the speedometer correlate with the current level of cognitive load?
Below we will state and explain our hypotheses for these questions. 

% ----- HYPOTHESES -----
Firstly, \citet{Palinko2010} expect that pupillometry can provide a viable estimation for cognitive load while driving. 
More specifically, they suggest a positive correlation between cognitive load and pupil size.
We therefore hypothesize that the current level of WML can be predicted by pupil size.

Secondly, results by \citet{Scheunemann2019} suggest an interaction between WML and visuospatial demands at the brain level. 
They conducted a driving experiment with functional near-infrared spectroscopy (fNIRS).
Results showed that the level of WML had a significant influence on the change in activation patterns in the brain associated with an increase in visuospatial demands.
This interaction effect is expected to be reflected in pupil size as well.

Thirdly, a hypothesized negative correlation between mental effort and fixations on the speedometer follows from the notion of limited cognitive resources (as described by \citet{DeWaard1996}).
For example, an easy driving task requires little cognitive resources to be spent.
This then leaves plenty of cognitive resources for checking the speedometer.
In contrast, a driving task with high visuospatial or central demands allows little ``mental space'' to concern oneself with the speedometer.
Findings by \citet{Salvucci2011} support this claim, suggesting that drivers perform less control updates (such as checking the speedometer) when engaging in a demanding secondary task.

% OPERATIONALISATION
In order to test these three hypotheses a simulated-driving experiment was conducted.
It largely follows the approach of \citet{Scheunemann2019} who studied the interaction between WML and visuospatial demands while driving.
Their experiment involved participants driving a car on a highway in a realistic simulation. 

% working memory load
WML was manipulated through an \nback task, which is considered to be a standard measure of working memory in cognitive neuroscience \citep{Kane2007}.
The task was integrated into the driving process by means of speed regulation, meaning participants were instructed to drive according to the speed sign that occurred \(n\) signs ago.

% visuospatial demands
Visuospatial demands were manipulated by contrasting two driving environments: \textit{construction} and \textit{non-construction}.
In the non-construction condition participants drove on a regular three-lane highway.
In the construction condition the leftmost lane was closed off by a continuous row of pylons.
Further, the remaining two lanes were of reduced width, which increases driving difficulty \citep{Liu2016}.