% !TEX root = thesis.tex

\section{Introduction}\label{sec:introduction}

% ----- GENERAL EXPLANATIONS -----
% driving
Driving a car is a challenging task. 
It involves processing a large amount of stimuli and constantly updating a mental model of the environment.
Not to mention, operating a vehicle requires making appropriate decisions to ensure the safety of both the driver and the drivers around them.
Taking all this into consideration it should not come as a surprise that human failure is the cause of the majority of traffic accidents \citep{DeWaard1996}.

% adaptive driving
An often proposed solution to this problem is automated driving \citep{Cabrall2018}.
However, \textit{fully} automated driving might not be desirable since the driver is kept ``out of the loop''.
Situations where they must suddenly take back manual control then pose a serious risk due to the reduced attentional awareness of the driver \citep{Dijksterhuis2012}.
The solution to this problem could then be adaptive automation.



% cognitive load

% visuospatial demands

% ways of measuring cognitive load: pupil dilation

% ----- RESEARCH QUESTIONS -----
This study aims to investigate three questions concerning working memory load (WML) while driving. 
Firstly, can the current level of WML be predicted by pupil dilation (1)? 
Secondly, is there an interaction between WML and visuospatial demands on pupil dilation (2)?
And finally, does the frequency of saccades to the speedometer correlate with the current level of mental effort, i.e.\@ WML and visuospatial demands (3)?

% ----- HYPOTHESES -----
Pupil dilation is expected to be a successful predictor for WML (1) since it is positively correlated with arousal \citep{Mathot2018} and therefore with mental effort.
In the context of the current experiment this hypothesis would manifest itself as a high \(n\)-back level corresponding to a large pupil dilation.

Next, results by \citet{Scheunemann2019} suggest an interaction between WML and visuospatial demands at the brain level. 
They conducted a driving experiment with functional near-infrared spectroscopy (fNIRS).
They found that the WML level had a significant influence on the change in activation patterns in the brain associated with an increase in visuospatial demands.
This interaction effect is expected to be reflected in pupil dilation as well (2).

Lastly, a hypothesized negative correlation between mental effort and saccades to the speedometer (3) follows from the notion of limited cognitive resources (as described by \citet{DeWaard1996}).
For example, an easy driving task requires little cognitive resources to be spent.
This then leaves plenty of cognitive resources for checking the speedometer.
In contrast, a driving task that is highly demanding (either visuospatially or mnemonically) allows little ``mental space'' to concern oneself with the speedometer.

% OPERATIONALISATION
In order to test these three hypotheses a simulated driving experiment was conducted.
It largely follows the approach of \citet{Scheunemann2019} who studied the interaction between WML and visuospatial demands while driving.
Their experiment involved participants driving a car on a highway in a realistic simulation. 

% working memory load
WML was manipulated through an \(n\)-back task, which is considered to be a standard measure of working memory in cognitive neuroscience \citep{Kane2007}.
The task was integrated into the driving process by means of speed regulation, meaning participants were instructed to drive according to the speed sign that occurred \(n\) signs ago.

% visuospatial demands
Visuospatial demands were manipulated by contrasting two driving environments: \textit{construction} and \textit{non-construction}.
In the non-construction condition participants drove on a regular three-lane highway.
In the construction condition the leftmost lane was closed off by a continuous row of pylons.
Further, the remaining two lanes were of reduced width, increasing driving difficulty.