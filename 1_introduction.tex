% !TEX root = thesis.tex

\section{Introduction}\label{sec:introduction}

% ----- GENERAL EXPLANATIONS -----
% driving
Driving a car is a challenging task. 
It involves processing a large amount of stimuli and constantly updating a mental model of the environment.
Not to mention, operating a vehicle requires making appropriate decisions to ensure the safety of both the driver and other road users.
Driving is even more challenging for young, novice drivers. 
They are more prone to a high level of mental workload than experienced drivers due to their low level of operating skills, lack of driving experience \citep{Gregersen1996} and not fully-matured prefrontal cortex \citep{Ross2014}.
This in turn is one of the causes for the relatively large number of traffic accidents that young drivers are involved in \citep{Sena2013}.
Taking all this into consideration, it should not come as a surprise that human failure is the cause of the majority of traffic accidents \citep{DeWaard1996}.

% adaptive driving
An often proposed solution to this problem is automated driving \citep{Cabrall2018}.
However, \textit{fully} automated driving might not be desirable since the driver is kept out of the loop.
Situations where they must suddenly take back manual control then pose a serious risk due to the reduced attentional awareness of the driver \citep{Dijksterhuis2012}.
The solution to this problem could then be adaptive automation.

In adaptive automation the division of control between the machine and the human operator is not static.
Rather, it is based on changes in the physical environment or the condition of the operator \citep{Sheridan2011}.
An important facet of this is adaptive automation based on the human factor of mental workload.
An intelligent car that can counteract the negative effects of a high cognitive load on driving performance can greatly benefit the driver's safety \citep{Hancock1988}.
As an example of automation within driving consider lane-keeping.
Though it may seem to be an easy part of driving, many road accidents are caused by a failure in lane-keeping \citep{Dijksterhuis2012}.
The applications of automated lane-keeping assistance range from a low-level aid such as visual feedback on the car's lateral position \citep{Dijksterhuis2012} to
a high-level aid such as the torque of the steering wheel being adapted to lane markings and the individual's driving style \citep{Rath2019}.

% cognitive load
When it comes to adaptive automation based on the cognitive load of the driver an important aspect is of course measuring cognitive load.
\citet{Haapalainen2010} define cognitive load as ``the level of perceived effort for learning, thinking and reasoning as an indicator of pressure on working memory'' (p.~302).
Changes in cognitive load have an effect on a number of physiological measures including but not limited to heart rate variability, brainwave levels (electroencephalogram; EEG), skin galvanic response and pupillary response \citep{Haapalainen2010}.
The current study focuses on the latter.

[To be added: something about visuospatial demands and their possible relation to cognitive load and therefore pupillometry.]

% ----- RESEARCH QUESTIONS -----
This study aims to investigate three questions concerning eye-tracking measurements and working memory load (WML) while driving. 
Firstly, can the current level of WML be predicted by pupil dilation (1)? 
Secondly, is there an interaction between WML and visuospatial demands on pupil dilation (2)?
And finally, does the frequency of fixations on the speedometer correlate with the current level of mental effort, i.e.\@ WML and visuospatial demands (3)?

% ----- HYPOTHESES -----
\citet{Palinko2010} expect that pupillometry can provide a viable estimation for cognitive load while driving (1). 
More specifically they suggest a positive correlation between pupil dilation and cognitive load.
In the context of the current experiment this hypothesis would manifest itself as a high \(n\)-back level corresponding to a large pupil dilation.

Next, results by \citet{Scheunemann2019} suggest an interaction between WML and visuospatial demands at the brain level. 
They conducted a driving experiment with functional near-infrared spectroscopy (fNIRS).
They found that the WML level had a significant influence on the change in activation patterns in the brain associated with an increase in visuospatial demands.
This interaction effect is expected to be reflected in pupil dilation as well (2).

Lastly, a hypothesized negative correlation between mental effort and fixations on the speedometer (3) follows from the notion of limited cognitive resources (as described by \citet{DeWaard1996}).
For example, an easy driving task requires little cognitive resources to be spent.
This then leaves plenty of cognitive resources for checking the speedometer.
In contrast, a driving task that is highly demanding (either visuospatially or mnemonically) allows little ``mental space'' to concern oneself with the speedometer.

% OPERATIONALISATION
In order to test these three hypotheses a simulated driving experiment was conducted.
It largely follows the approach of \citet{Scheunemann2019} who studied the interaction between WML and visuospatial demands while driving.
Their experiment involved participants driving a car on a highway in a realistic simulation. 

% working memory load
WML was manipulated through an \(n\)-back task, which is considered to be a standard measure of working memory in cognitive neuroscience \citep{Kane2007}.
The task was integrated into the driving process by means of speed regulation, meaning participants were instructed to drive according to the speed sign that occurred \(n\) signs ago.

% visuospatial demands
Visuospatial demands were manipulated by contrasting two driving environments: \textit{construction} and \textit{non-construction}.
In the non-construction condition participants drove on a regular three-lane highway.
In the construction condition the leftmost lane was closed off by a continuous row of pylons.
Further, the remaining two lanes were of reduced width, increasing driving difficulty.