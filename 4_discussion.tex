% !TEX root = thesis.tex

\section{Discussion}\label{sec:discussion}
In this study we sought to answer three questions related to cognitive load while driving by means of an eye-tracking experiment.
\textbf{(1)} Can the current level of working memory load (WML) be predicted by pupil size?
\textbf{(2)} Is there an interaction between WML and visuospatial demands on pupil size?
\textbf{(3)} Does the frequency of fixations on the speedometer correlate with the current level of cognitive load?
Below you will find our conclusions based on the results of the experiment.

Let us first consider WML and pupil size within a trial.
As we saw before, there is a significant effect of WML and speed sign number on pupil size (see Figure~\ref{fig:ps-speed-sign}).
Most notably we see a contrast between pupil size for low \nback levels (\(n = 0,1,2\)) and high \nback levels (\(n = 3,4\)). 
This contrast between low- and high-level \nback levels is also reflected in task error (see Figure~?, Milan's task error plot).
We can therefore conclude that pupil size is a predictor for task difficulty and by extension, working memory load. 

Interestingly, compared to \(n = 3\) the peak in pupil size for \(n = 4\) is higher but shows a sharper decline afterwards.
An explanation for this could be that participants focus their attention on the task at first (causing their pupil to dilate), but then quickly abandon the task because of its difficulty, resulting in a contraction of the pupil.

For the mean pupil size over a trial we also found a significant effect of \nback level, confirming our hypothesis that pupil size is a predictor for WML.\@

Next, we expected an interaction between WML and visuospatial demands on pupil size based on results by \citet{Scheunemann2019}.
However, this hypothesis was not confirmed by the results of this experiment. 
Our results suggest that visuospatial demands have little influence on pupil size.
One could take this to mean that pupil size is an indication of working memory load only.

Finally, results showed that the amount of fixations on the speedometer decreases linearly by WML, confirming our hypothesis.
We conclude that as WML increases, fewer cognitive resources are available for performing control updates.
This could indicate that high WML leads to a decline in speed-keeping.

There are some limitations to this study. 
Firstly, something something.