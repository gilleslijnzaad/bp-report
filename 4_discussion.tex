% !TEX root = thesis.tex

\section{Discussion}\label{sec:discussion}
In this study we sought to answer two questions related to cognitive load while driving by means of an eye-tracking experiment.
Firstly, can the current level of cognitive load be predicted by pupil size? 
And secondly, does the frequency of fixations on the speedometer correlate with the current level of cognitive load?
Below you will find our conclusions based on the results of the experiment.

Let us first consider the effect of working memory load (WML) on pupil size.
Within trials we found a significant effect of WML and speed sign number on pupil size (see Figure~\ref{fig:ps-speed-sign}).
Most notably we see a contrast between pupil size for low \nback levels (\(n = 0,1,2\)) and high \nback levels (\(n = 3,4\)). 
This contrast between low- and high-level \nback levels is also reflected in task error (see Figure~?, Milan's task error plot). 

Interestingly, compared to \(n = 3\) the peak in pupil size for \(n = 4\) is higher but shows a sharper decline afterwards.
An explanation for this could be that participants focus their attention on the task at first (causing their pupil to dilate), but then quickly abandon the task because of its difficulty, resulting in a contraction of the pupil.

Between trials we also found a significant effect of WML on pupil size. 
We therefore conclude that pupil size is a predictor for WML.\@

However, the same cannot be said for visuospatial demands. No effect of construction condition on pupil size was found, and the expected interaction between WML and visuospatial demands on pupil size was not significant.
Our results therefore suggest that visuospatial demands have little influence on pupil size.
One could take this to mean that changes in pupil size \textit{only} reflect working memory load.

Finally, results showed that the amount of fixations on the speedometer decreases linearly by WML.\@
We conclude that as WML increases, fewer cognitive resources are available for performing control updates.
This could indicate that high WML leads to decreasing speed-keeping efforts.
In this context we again see little to no influence of visuospatial demands, suggesting that only WML affects speed-keeping efforts.

What are then the practical implications of these results? 
We have established that pupillometry can be used to assess working memory load (WML) during driving, yet reveals little about visuospatial demands on the driver.
Since a robust measure of cognitive load must embody both central \textit{and} visual demands, our results challenge the utility of pupillometry in the estimation of drivers' cognitive load.
Further research on the relation between visuospatial demands and pupil size during driving is therefore necessary.
If our findings are further consolidated, pupil size should no longer be a variable of interest in the development of adaptive driving based on cognitive load.

Moreover, the confirmed correlation between WML and speed-keeping efforts highlights speed-keeping as a topic of interest in the field of adaptive automation.
It is important to then first investigate the relationship between speed-keeping efforts and actual performance.

There are some limitations to this study. 
Firstly, for 10 of the 38 participants, trials were shorter than they should have been due to an error in the simulation program.
More specifically, only 9 speed signs were shown for all trials. 
This is the correct number of signs for only the 0-back task.
For all other \nback levels there were \(n\) signs too little. 
This is especially important for the 4-back, where the time spent performing the speed regulation task is essentially halved.
These 10 participants only performed 82\% of the entire experiment.
This problem diminishes the statistical power of our results somewhat.

[I cannot think of any more limitations. Maybe we can discuss it in our meeting this Tuesday.]